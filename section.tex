\documentclass[12pt]{article}
\usepackage[fontset=ubuntu]{ctex}

\begin{document}

\newcommand{\quizitem}{\underline{\hspace{.1\textwidth}}}
\newcommand{\inspic}[1]{
\vskip -.1\textheight \hskip .6\textwidth
\includegraphics[width=0.2\textwidth]{#1}
}

\newcounter{question}
\newcommand{\quiz}[5]{
\addtocounter{question}{1}
   \begin{block}{\color{red}\arabic{question}}
       #1
   \begin{enumerate}[A]
      \item #2
      \item #3
      \item #4
      \item #5
  \end{enumerate}
  \end{block}
}

\begin{frame}{}
    \centering\color{red}
    \Huge  QUIZ
\end{frame}

\begin{frame}{}
\quiz{补码表示的16位带符号数字8088H除以8,结果为\quizitem 。}
{1011H}
{8011H}
{F088H}
{F011H}
\inspic{accessories-calculator.png}
\end{frame}

\begin{frame}{}
\quiz{设某计算机用双字节表示一个浮点数,其中阶符、阶码8位,用原码表示;
   数符(尾符)、尾数8位,用补码表示,则能表示的最大值和最小非零值
    的比值为\quizitem 。(考虑非规格化)}
{$2^{261}$}
{$2^{256}$}
{$2^{254}$}
{$2^{128}$}
\inspic{format-text-direction-ltr.png}
\end{frame}

\begin{frame}{}
\quiz{以下改变标志寄存器的指令是\quizitem 。}
{XCHG AX,BX}
{PUSHF}
{IN AL,DX}
{SUB AX,AX}
\inspic{applications-utilities.png}
\end{frame}

\begin{frame}{}
\quiz{在8086/8088系统中,一个基本总线周期由4个时钟周期(T状态)组成。
     在T1状态,总线产生\quizitem 信息。}
{数据}
{状态}
{地址}
{其它}
\inspic{tux.png}
\end{frame}

\begin{frame}{}
\quiz{8086 CPU上电复位后,第一条指令的地址在\quizitem 。}
{00000H}
{FFFF0H}
{FFFFFH}
{F0000H}
\inspic{battery-good-charging.png}
\end{frame}

%\begin{frame}{}
%\quiz{若容量为512K位(bit)SRAM芯片有4条数据线,则它的地址线数目为\quizitem 。}
%{15}{16}{17}{18}
%\vskip -.1\textheight \hskip .6\textwidth
%\includegraphics[width=0.2\textwidth]{accessories-calculator.png}
%\end{frame}

\begin{frame}{}
\quiz{地址/数据总线复用的情况下,存储器地址线和CPU地址线引脚\quizitem 。}
{需通过总线控制器相连}
{需通过地址锁存器相连}
{需通过数据收发器相连}
{可以直接相连}
\inspic{preferences-system-privacy.png}
\end{frame}

\begin{frame}{}
\quiz{与动态存储器相比,静态存储器具有\quizitem 优势。}
{单位成本低}
{抗掉电}
{速度快}
{使用寿命长}
\inspic{preferences-desktop-theme.png}
\end{frame}

\begin{frame}{}
\quiz{8片8K $\times$ 8bit 的存储器组成 32 位数据总线系统的 64KB
连续存储空间,用于片选的地址线最少有\quizitem 根。}
{1}
{2}
{3}
{4}
\inspic{emblem-favorite.png}
\end{frame}

\begin{frame}{}
\quiz{高速缓存通常由\quizitem 构成。}
{硬盘的一部分}
{主存的一部分}
{静态存储器}
{动态存储器}
\inspic{pitivi.png}
\end{frame}

\begin{frame}{}
\quiz{内存和外存的技术特点主要体现在\quizitem 上。}
{速度}
{易失性}
{存储原理}
{与CPU的连接方式}
\inspic{totem.png}
\end{frame}

\begin{frame}{}
   \begin{block}{参考答案}
           \centering

   \begin{tabbing}
      \hspace{.1\textwidth}\=
      \hspace{.1\textwidth}\=
      \hspace{.1\textwidth}\=
      \hspace{.1\textwidth}\=
      \hspace{.1\textwidth}\=
      \hspace{.1\textwidth}\=\kill
      \> D \> A \> D \> C \> B \\
      \> B \> C \> A \> C \> D
    \end{tabbing}
   \end{block}
\end{frame}

\begin{frame}{}
\quiz{将一个字节变量 X 的低5位清零,其他位保持不变,正确的操作是\quizitem 。}
    {OR  X, 1110 0000B}
    {XOR X, 1110 0000B}
    {XOR X, 0001 1111B}
    {AND X, 1110 0000B}
\inspic{scinotes.png}
\end{frame}

\begin{frame}{}
\quiz{一篇10万字的中文中篇小说,以UTF-8编码保存的文件大小大约是\quizitem 。}
{10KB}{100KB}{200KB}{300KB}
\inspic{xcos.png}
\end{frame}

\begin{frame}{}
\quiz{8086中,与标志位IF相关的引脚是\quizitem 。}
    {$\overline{\rm INTA}$}{INTR}{HOLD}{NMI}
\inspic{windowslogo.png}
\end{frame}

\begin{frame}{}
\quiz{某存储器外部引脚标有 A0$\sim$A12、 D0$\sim$D7、 RAS、 CAS及
    读/写、片选等,它的最大可能容量是\quizitem 。}
    {8KB}{16KB}{16MB}{64MB}
\inspic{Thunar.png}
\end{frame}

\begin{frame}{}
\quiz{Pentium处理器采用了超标量流水线技术。所谓超标量是指\quizitem 。}
    {两条或两条以上指令流水线}
    {四条或四条以上指令流水线}
    {五级以上流水线深度}
	{十级以上流水线深度}
\inspic{vlc-xmas.png}
\end{frame}

\begin{frame}{}
\quiz{16位处理器 8086 读写一个连续的双字节数据,总线周期数 \quizitem 。}
{只需一个}{必须两个}
{可能一个也可能两个}{取决于存储器的速度}
\inspic{audio-headset.png}
\end{frame}

\begin{frame}{}
\quiz{以下与其他三个读写特性不同的存储设备是 \quizitem 。}
{TF/SD卡}{U盘}{固态硬盘SSD}{内存条}
\inspic{format-text-direction-ltr.png}
\end{frame}

\begin{frame}{}
\quiz{高速缓存可以加快程序运行速度的机理在于 \quizitem 。}
{可以缩短CPU访问存储器的时间}
{能提高主存的工作频率}
{能提高CPU的工作频率}
{增加了存储器的容量}
\inspic{emixer.png}
\end{frame}

\begin{frame}{}
\quiz{分页存储器管理技术中通常采用多级页表,其目的是\quizitem 。}
{减少页面占用的物理内存}
{减少页表占用的物理内存}
{可以灵活改变页面大小}{简化查表过程}
\inspic{preferences-desktop-theme.png}
\end{frame}

\begin{frame}{}
\quiz{根据总线上传输的信息不同,将总线分为数据总线、地址总线、控制总线三组。
   以下关于总线信号传输方向的叙述,正确的是 \quizitem 。}
{地址总线单向输出,控制总线既有输入也有输出}
{数据总线单向,控制总线双向}
{地址总线为双向三态}{控制总线为单向输出}
\inspic{battery-good-charging.png}
\end{frame}

\begin{frame}{}
   \begin{block}{参考答案}
           \centering

   \begin{tabbing}
      \hspace{.1\textwidth}\=
      \hspace{.1\textwidth}\=
      \hspace{.1\textwidth}\=
      \hspace{.1\textwidth}\=
      \hspace{.1\textwidth}\=
      \hspace{.1\textwidth}\=\kill
      \> D \> D \> B \> D \> A \\
      \> C \> D \> A \> B \> A
    \end{tabbing}
   \end{block}
\end{frame}

\begin{frame}{}
\quiz{对于两个不相等非零数的三种位逻辑操作 (与、或、异或),以下正确的论断是
	\quizitem 。}
{与操作的结果可能为0}
{或操作的结果可能为0}
{异或操作的结果可能为0}
{或的结果,真值不小于其中的任何一个(补码)}
\end{frame}

\begin{frame}{}
\quiz{CPU 在调用子程序时,\quizitem 会被压入堆栈。}
 {目标地址}{返回地址}{标志寄存器}{程序计数器}
\end{frame}

\begin{frame}{}
\quiz{汇编语言中,伪指令的作用是 \quizitem。}
    {用于简化机器指令的书写}{用于指导编译器编译}
    {用于帮助程序员组织源程序}{用于优化机器码}
\end{frame}

\begin{frame}{}
\quiz{通用计算机系统中,作为大容量主存使用的存储设备是\quizitem。}
    {SRAM}{SDRAM}{EEPROM}{SSD}
\end{frame}

\begin{frame}{}
\quiz{以下 \quizitem 的处理器具有指令集兼容性。}
    {Intel X86 与 AMD}{Intel IA32 与 IA64}
    {Intel处理器 与 ARM 处理器}{Intel IA64 与 64位 ARMv8}
\end{frame}

\begin{frame}{}
\quiz{两条顺序执行的指令,能发挥超标量流水线优势的程序片是\quizitem 。}
{\parbox[t]{3cm}{R0 = R1 + R2\\ R1 = R1 + R0\\~}}
{\parbox[t]{3cm}{R0 = R1 * R2\\ R1 = R1 + R0\\~}}
{\parbox[t]{3cm}{R0 = R1 * R2\\ R2 = R1 + R2\\~}}
{\parbox[t]{3cm}{R0 = R1 + R2\\ R0 = R0 + R0}}
\end{frame}

\begin{frame}{}
\quiz{为 x86 保护方式设计的程序不能在8086中运行,其原因是\quizitem 。}
{安全性得不到保障}{8086主频不够或内存不够}{指令集不兼容}{寻址方式不兼容}
\end{frame}

\begin{frame}{}
\quiz{Intel 处理器实方式和保护方式的本质区别是\quizitem 。}
{不同的任务管理机制}{支持不同的指令集}
{存储器管理方式的不同}{使用不同的寄存器组}
\end{frame}

\begin{frame}{}
\quiz{微机系统中,存储器容量的合理配置是\quizitem 。}
{静态存储器 $<$ 动态存储器}{动态存储器 $<$ 静态存储器}
{只读存储器 $<$ 随机存储器}{随机存储器 $<$ 只读存储器}
\end{frame}

\begin{frame}{}
\quiz{虚拟存储系统由``主存---辅存''两级存储器组成,其作用是解决\quizitem
    的问题。}
{主存容量不足}{主存与辅存速度不匹配}
{辅存与CPU速度不匹配}{主存与CPU速度不匹配}
\end{frame}

\begin{frame}{}
   \begin{block}{参考答案}
           \centering

   \begin{tabbing}
      \hspace{.1\textwidth}\=
      \hspace{.1\textwidth}\=
      \hspace{.1\textwidth}\=
      \hspace{.1\textwidth}\=
      \hspace{.1\textwidth}\=
      \hspace{.1\textwidth}\=\kill
      \> A \> B \> B \> B \> A \\
      \> C \> D \> C \> A \> A
    \end{tabbing}
   \end{block}
\end{frame}
\end{document}
